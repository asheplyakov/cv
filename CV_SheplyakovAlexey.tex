\documentclass{report}
\usepackage[utf8x]{inputenc}
\usepackage[english]{babel}
\usepackage{savetrees}
\usepackage{hyperref}

\newcommand{\Obig}[1]{\ensuremath{\mathcal{O}(#1)}}
\pdfminorversion 3%

\author{Alexey A. Sheplyakov}
\title{Alexey A. Sheplyakov. Curriculum vitae}
\begin{document}
\pagestyle{empty}
\noindent{\bf Name:} Alexey A. Sheplyakov \\
{\bf Contact info:} 
  {\bf email:} \href{mailto:asheplyakov@yandex.ru}{asheplyakov@yandex.ru} %
, {\bf phone:} +7 987 813 92 79 \\
{\bf Date of birth:} 19 April, 1978 \\
{\bf Place of birth:} Kharkov region, USSR \\
{\bf Nationality:} Russian Federation \\
{\bf Sex:} male \\
{\bf Languages:} English (fluent), Russian (mother tongue), Ukrainian (mother tongue)

\paragraph{Education}
\begin{itemize}
\item
October 2001 -- October 2004: postgraduate student at
\href{http://newuc.jinr.ru}{University Center} of
\href{http://www.jinr.ru}{Joint Institute for Nuclear Research}
\item
September 1995 -- March 2001: graduate student at
\href{http://www-htuni.univer.kharkov.ua/ftf/index.htm}{Department of Physics and Technology}
of
\href{http://www.univer.kharkov.ua/en}{V.N.~Karazin Kharkov National University}
\end{itemize}

\paragraph{Work experience}
\begin{itemize}
\item
  {\bf October 2019 -- now:} C developer at \href{http://basealt.ru}{BaseALT}\\
  Responsibilities:
   \begin{itemize}
     \item Porting Linux (kernel) to various ARM and MIPS SoCs
     \item Debugging and fixing issues in vendor drivers and firmware
   \end{itemize}
\item
  {\bf November 2018 -- September 2019:} performance analyst at \href{http://epam.com}{EPAM}\\
  Performance analysis and improvement of "low latency" code (software for capital markets).
  Responsibilities:
   \begin{itemize}
      \item Designing benchmark scenarios based on real world products usage
      \item Choosing relevant metrics and visualization
      \item Writing load generators and other test applications, automating benchmarks
      \item Identifying bottlenecks, improving the architecture and implementation
      \item Tuning the OS (Linux) for low latency/soft real time applications
   \end{itemize}

  Projects:
  \begin{itemize}
    \item \href{https://www.b2bits.com/trading_solutions/fixedge.html}{FIXEdge}, application server providing
          \href{https://en.wikipedia.org/wiki/Financial_Information_eXchange}{FIX} (Financial Information eXchange)
          gateway to multiple protocols (various message queues, REST, etc). \\
          Technologies: C++11, Berkeley sockets, boost, Python, ansible, CMake
  \end{itemize}

\item
  {\bf May 2018 -- November 2018:} C developer at \href{http://www.basealt.ru}{BaseALT}\\
  Responsibilities:
  \begin{itemize}
  \item Debugging, fixing, and improving Linux' client side AD (Active Directory)
        integration software, such as \href{https://pagure.io/SSSD}{SSSD}
  \item Automation of \href{http://www.samba.org}{Samba} based Active Directory
	domains deployment
  \end{itemize}
\item
  {\bf August 2017 -- May 2018:} performance analyst at \href{http://epam.com}{EPAM} \\
\item 
  {\bf August 2013 -- July 2017:} C++/C/Python developer at \href{https://www.mirantis.com}{Mirantis, Inc.} \\
 Projects:
 \begin{itemize}
   \item \href{http://ceph.com}{Ceph}, a distributed object storage, block device,
     and a POSIX filesystem. \\
     Responsibilities:
     \begin{itemize}
       \item Debugging and bugfixing (examples:
	  \href{https://github.com/ceph/ceph/commit/aba6746b850e9397ff25570f08d0af8847a7162c}{ceph bug \#14428},
	  \href{https://github.com/ceph/ceph/commit/918c12c2ab5d014d0623b1accf959b041aac5128}{ceph bug \#12065},
	  \href{https://github.com/ceph/ceph/commit/aab3a40f95dafab34a7eadc2159d142a5f0f88f1}{ceph bug \#14512})
       \item Troubleshooting production Ceph clusters
       \item Planning and deployment of Ceph clusters according to clients' workload and requirements
     \end{itemize}
     Technologies: C++11, boost, Python, ansible

   \item
   \href{https://www.mirantis.com/software/mirantis-openstack-software}{Mirantis OpenStack} \\
  Responsibilities:
  \begin{itemize}
     \item
       Debugging and fixing race conditions, data corruption, hardware specific,
       and other "interesting" bugs
       (\href{http://stackalytics.com/?release=all&company=mirantis&user_id=asheplyakov&metric=commits}{commits})
     \item
       Adjusting OpenStack components according to customers' requirements
     \item
       \href{http://stackalytics.com/?release=all&company=mirantis&user_id=asheplyakov&metric=marks}{Reviewing the code}
  \end{itemize}
  Technologies: Python, puppet, POSIX shell, ruby, GNU make
  \end{itemize}

\item
  {\bf November 2012 -- August 2013}: C developer at
  \href{http://www.adbglobal.com}{ADB Ukraine} \\
  Project:
  {Epicentro~Platform}, CPE (home router) firmware based on embedded Linux. \\
  Responsibilities:
  \begin{itemize}
     \item Implementation of user space IPv6 related components (DHCPv6 server and client)
     \item Bug fixing (mostly the userspace code)
   \end{itemize}
  Technologies: C, Berkeley sockets, Linux, IPv6

\item {\bf April 2011 -- November 2012}: C++ team lead at
  \href{http://p-product.com}{P-Product, Inc.}
  Projects:
  \begin{itemize}
  \item
    \href{http://www.ceva-dsp.com}{CEVA} profiler,
    DSP (digital signal processor) profiler based on a simulator of
    the target architecture, a part of
    \href{http://www.ceva-dsp.com/products/tools/software/index.php}{CEVA IDE}
    Responsibilities:
    \begin{itemize}
      \item Design and implementation of raw performance samples processing
      \item Representation of the performance data in the GUI
    \end{itemize}
    Technologies: C++, Qt4 (in particular GraphicsScene framework), boost, SQLite

  \item
    \href{http://www.xmpie.com}{XMPie} uRender, video personalization solution.
    Responsibilities:
    \begin{itemize}
      \item Design and implementation of H264 video streams stitching algorithm
            (the key feature of the program)
      \item Design and implementation of the GUI
      \item Solving portability issues (the product runs on Mac OS X, Linux, and Windows)
    \end{itemize}
    Technologies: \href{http://ffmpeg.org}{libavcodec}, Qt4
  \end{itemize}

\item {\bf October 2010 -- April 2011:} software engineer at
  \href{http://www.heliconsoft.com}{Helicon Soft Ltd.} \\
  Project:
  Helicon Remote 2.0, a program for semi-atomatic focus and exposure bracketing.\\
  Responsibilities:
     \begin{itemize}
       \item Writing device (photo camera) communication code for Mac OS X and Linux
       \item Design and implementation of GUI
       \item Reverse engineering undocumented cameras' commands
       \item Troubleshooting Mac OS X specific issues
    \end{itemize}
    Technologies: \href{http://www.libusb.org}{libusb}, libptp, Canon SDK, Qt4

\item {\bf February 2010 -- July 2010:} software engineer at Quickoffice, Inc \\
  Project: Qt4 port of Quickoffice mobile office suite  \\
  Responsibilities:
    \begin{itemize}
      \item Rewriting Symbian specific code the Qt4 way
      \item Debugging and fixing obscure bugs (like "widget gets mispositioned
            after rotating the phone 3 times in a row")
    \end{itemize}

\item {\bf November 2008 -- January 2010:} software developer at
  \href{http://www.metalika.ua}{Metalika Publishing House} \\
  Project:
    {\it Opus Metalicus} -- the program which powers the
    \href{http://www.metalika.ua/metallicheskii-vestnik.html}{``Metallicheskiy Vestnik''}
    magazine. \\
    Responsibilities:
    \begin{itemize}
      \item Design and implementation of the database schema, triggers, views
      \item Design and implementation of the GUI
      \item Deployment, administration, and support of the software
    \end{itemize}
    Technologies: Qt4, PostgreSQL

\item {\bf November 2004 -- November 2008:} research fellow at
  \href{http://theor.jinr.ru}{Bogoliubov Laboratory of Theoretical Physics}. \\
  Supervisor: professor~%
  \href{http://theor.jinr.ru/~kazakovd}{D.I. Kazakov} \\
  \paragraph{Research topics}
    \begin{itemize}
      \item Phenomenology of the Standard Model of elementary particles
	      and its minimal supersymmetric extension
      \item Application of computer algebra to calculations in perturbative
            quantum field theory, in particular computation of Feynman integrals
	    with several mass scales 
    \end{itemize}

   \paragraph{Publications}
    \begin{itemize}
      \item
        A. Bednyakov, D.I. Kazakov, A. Sheplyakov,
        ``On the 2-loop $\Obig{\alpha_s^2}$ corrections to the pole mass of the t
          quark in the MSSM.''
        Physics of Atomic Nuclei, 71:343-350,2008.
        (preprint: \href{http://arxiv.org/abs/hep-ph/0507139}{hep-ph/0507139})
      \item
        M.Yu. Kalmykov and A. Sheplyakov,
	``lsjk -- a C++ library for arbitrary-precision numeric evaluation of
          the generalized log-sine functions.''
        Comput.Phys.Commun. 172 (2005) 45-59.
	(preprint: \href{http://arxiv.org/abs/hep-ph/0411100}{hep-ph/0411100})
      \item
        A. Bednyakov and A. Sheplyakov,
	``Two-loop $\Obig{\alpha_s y^2}$ and $\Obig{y^4}$ MSSM corrections to
	  the pole mass of the $b$-quark.''
        Physics Letters B, 604:91-97,2004.
        (preprint: \href{http://arxiv.org/abs/hep-ph/0410128}{hep-ph/0410128})
    \end{itemize}

  \paragraph{Talks}
    \begin{itemize}
      \item
        The 15th International Conference on Supersymmetry and the Unification
	of Fundamental Interactions
	(\href{http://susy07.uni-karlsruhe.de}{SUSY07}), Karlsruhe, Germany,
	July 26 - August 1, 2007 \\
	Subject: ffmssmsc -- a C++ library for superpartner mass calculation and
	renormalization group analysis of the MSSM
        (\href{http://indico.cern.ch/contributionDisplay.py?contribId=419&sessionId=236&confId=6210}{slides})
      \item
        International school-workshop ``Calculations for modern and future colliders''
	(\href{http://theor.jinr.ru/~calc2006}{CALC-2006}),
	Dubna, Russia, July 15-25, 2006 \\
	Subject: Different two-loop correction to the mass of heavy quarks in the MSSM
	(\href{http://theor.jinr.ru/~calc2006/Talks/sheplyakov_cal06.pdf}{slides})
    \end{itemize}
\end{itemize}

\paragraph{Programming languages}
\begin{itemize}
\item
Primary (being used in everyday work): {\rm C++}, {\rm C}, Python,
\href{http://en.wikipedia.org/wiki/Bourne_shell}{Bourne shell}, SQL
\item
Others (used only occasionally):
Assembler (x86), ruby,
\href{http://en.wikipedia.org/wiki/Scheme_(programming_language)}{Scheme}
(mostly \href{http://www.gnu.org/software/guile}{guile})
\end{itemize}

\paragraph{Operating systems: }
Linux (primary OS since Oct. 1996), Mac OS X

\paragraph{Degrees}
M.Sc., March 2001 \\
	Supervisor: assistant professor
\href{http://www.slac.stanford.edu/spires/find/hep/www?rawcmd=FIND+A+ZIMA%2CV.G.&FORMAT=www&SEQUENCE=ds%28d%29}{\underline{V.G.~Zima}} \\
	Diploma thesis title: BFV-BRST quantization of free massless
	arbitrary spin particle.

\paragraph{Hobbies}
\begin{itemize}
\item Phenomenology the Standard Model and its supersymmetric extensions.
\item Calculation of Feynman integrals with several mass scales.
\item Application of computer algebra to calculations in perturbative quantum field theory.
\item Non-perturbative quantum field theory.
\item \href{http://www.ginac.de}{GiNaC}, a special purpose C++ library for
  symbolic computations.
\end{itemize}

\end{document}

